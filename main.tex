%% Run LaTeX on this file several times to get Table of Contents,
%% cross-references, and citations.

\documentclass[11pt]{book}
\usepackage{gvv}
\usepackage{gvv-book-bkup}
%\usepackage{Wiley-AuthoringTemplate}
\usepackage[sectionbib,authoryear]{natbib}% for name-date citation comment the below line
%\usepackage[sectionbib,numbers]{natbib}% for numbered citation comment the above line

%%********************************************************************%%
%%       How many levels of section head would you like numbered?     %%
%% 0= no section numbers, 1= section, 2= subsection, 3= subsubsection %%
\setcounter{secnumdepth}{3}
%%********************************************************************%%
%%**********************************************************************%%
%%     How many levels of section head would you like to appear in the  %%
%%				Table of Contents?			%%
%% 0= chapter, 1= section, 2= subsection, 3= subsubsection titles.	%%
\setcounter{tocdepth}{2}
%%**********************************************************************%%
\setcounter{tocdepth}{3}
%\includeonly{ch01}
\makeindex

\begin{document}

\frontmatter
%%%%%%%%%%%%%%%%%%%%%%%%%%%%%%%%%%%%%%%%%%%%%%%%%%%%%%%%%%%%%%%%
%% Title Pages
%% Wiley will provide title and copyright page, but you can make
%% your own titlepages if you'd like anyway
%% Setting up title pages, type in the appropriate names here:

\booktitle{CBSE Math}

\subtitle{Made Simple}

\AuAff{G. V. V. Sharma}


%% \\ will start a new line.
%% You may add \affil{} for affiliation, ie,
%\authors{Robert M. Groves\\
%\affil{Universitat de les Illes Balears}
%Floyd J. Fowler, Jr.\\
%\affil{University of New Mexico}
%}

%% Print Half Title and Title Page:
%\halftitlepage
\titlepage

%%%%%%%%%%%%%%%%%%%%%%%%%%%%%%%%%%%%%%%%%%%%%%%%%%%%%%%%%%%%%%%%
%% Copyright Page

\begin{copyrightpage}{2023}
%Title, etc
\end{copyrightpage}

% Note, you must use \ to start indented lines, ie,
% 
% \begin{copyrightpage}{2004}
% Survey Methodology / Robert M. Groves . . . [et al.].
% \       p. cm.---(Wiley series in survey methodology)
% \    ``Wiley-Interscience."
% \    Includes bibliographical references and index.
% \    ISBN 0-471-48348-6 (pbk.)
% \    1. Surveys---Methodology.  2. Social 
% \  sciences---Research---Statistical methods.  I. Groves, Robert M.  II. %
% Series.\\

% HA31.2.S873 2004
% 001.4'33---dc22                                             2004044064
% \end{copyrightpage}

%%%%%%%%%%%%%%%%%%%%%%%%%%%%%%%%%%%%%%%%%%%%%%%%%%%%%%%%%%%%%%%%
%% Only Dedication (optional) 

%\dedication{To my parents}

\tableofcontents

%\listoffigures %optional
%\listoftables  %optional

%% or Contributor Page for edited books
%% before \tableofcontents

%%%%%%%%%%%%%%%%%%%%%%%%%%%%%%%%%%%%%%%%%%%%%%%%%%%%%%%%%%%%%%%%
%  Contributors Page for Edited Book
%%%%%%%%%%%%%%%%%%%%%%%%%%%%%%%%%%%%%%%%%%%%%%%%%%%%%%%%%%%%%%%%

% If your book has chapters written by different authors,
% you'll need a Contributors page.

% Use \begin{contributors}...\end{contributors} and
% then enter each author with the \name{} command, followed
% by the affiliation information.

% \begin{contributors}
% \name{Masayki Abe,} Fujitsu Laboratories Ltd., Fujitsu Limited, Atsugi, Japan
%
% \name{L. A. Akers,} Center for Solid State Electronics Research, Arizona State University, Tempe, Arizona
%
% \name{G. H. Bernstein,} Department of Electrical and Computer Engineering, University of Notre Dame, Notre Dame, South Bend, Indiana; formerly of
% Center for Solid State Electronics Research, Arizona
% State University, Tempe, Arizona 
% \end{contributors}

%%%%%%%%%%%%%%%%%%%%%%%%%%%%%%%%%%%%%%%%%%%%%%%%%%%%%%%%%%%%%%%%
% Optional Foreword:

%\begin{foreword}
%\lipsum[1-2]
%\end{foreword}

%%%%%%%%%%%%%%%%%%%%%%%%%%%%%%%%%%%%%%%%%%%%%%%%%%%%%%%%%%%%%%%%
% Optional Preface:

%\begin{preface}
%\lipsum[1-1]
%\prefaceauthor{}
%\where{place\\
% date}
%\end{preface}

% ie,
% \begin{preface}
% This is an example preface.
% \prefaceauthor{R. K. Watts}
% \where{Durham, North Carolina\\
% September, 2004}

%%%%%%%%%%%%%%%%%%%%%%%%%%%%%%%%%%%%%%%%%%%%%%%%%%%%%%%%%%%%%%%%
% Optional Acknowledgments:

%\acknowledgments
%\lipsum[1-2]
%\authorinitials{I. R. S.}  

%%%%%%%%%%%%%%%%%%%%%%%%%%%%%%%%
%% Glossary Type of Environment:

% \begin{glossary}
% \term{<term>}{<description>}
% \end{glossary}

%%%%%%%%%%%%%%%%%%%%%%%%%%%%%%%%
%\begin{acronyms}
%\acro{ASTA}{Arrivals See Time Averages}
%\acro{BHCA}{Busy Hour Call Attempts}
%\acro{BR}{Bandwidth Reservation}
%\acro{b.u.}{bandwidth unit(s)}
%\acro{CAC}{Call / Connection Admission Control}
%\acro{CBP}{Call Blocking Probability(-ies)}
%\acro{CCS}{Centum Call Seconds}
%\acro{CDTM}{Connection Dependent Threshold Model}
%\acro{CS}{Complete Sharing}
%\acro{DiffServ}{Differentiated Services}
%\acro{EMLM}{Erlang Multirate Loss Model}
%\acro{erl}{The Erlang unit of traffic-load}
%\acro{FIFO}{First in - First out}
%\acro{GB}{Global balance}
%\acro{GoS}{Grade of Service}
%\acro{ICT}{Information and Communication Technology}
%\acro{IntServ}{Integrated Services}
%\acro{IP}{Internet Protocol}
%\acro{ITU-T}{International Telecommunication Unit -- Standardization sector}
%\acro{LB}{Local balance}
%\acro{LHS}{Left hand side}
%\acro{LIFO}{Last in - First out}
%\acro{MMPP}{Markov Modulated Poisson Process}
%\acro{MPLS}{Multiple Protocol Labeling Switching}
%\acro{MRM}{Multi-Retry Model}
%\acro{MTM}{Multi-Threshold Model}
%\acro{PASTA}{Poisson Arrivals See Time Averages}
%\acro{PDF}{Probability Distribution Function}
%\acro{pdf}{probability density function}
%\acro{PFS}{Product Form Solution}
%\acro{QoS}{Quality of Service}
%\acro{r.v.}{random variable(s)}
%\acro{RED}{random early detection}
%\acro{RHS}{Right hand side}
%\acro{RLA}{Reduced Load Approximation}
%\acro{SIRO}{service in random order}
%\acro{SRM}{Single-Retry Model}
%\acro{STM}{Single-Threshold Model}
%\acro{TCP}{Transport Control Protocol}
%\acro{TH}{Threshold(s)}
%\acro{UDP}{User Datagram Protocol}
%\end{acronyms}

\setcounter{page}{1}

\begin{introduction}
This book links high school coordinate geometry to linear algebra and matrix analysis through solved problems.

\end{introduction}

\mainmatter
\chapter{Vectors}
\section{2020}
\subsection{10}
\input{2020/vetors1.0.tex}
\subsection{12}
\input{2020/vetors2.0.tex}
\section{2023}
\subsection{10}
\input{2023/vectors10-1.tex}
\subsection{10}
\input{2023/Vectors.tex}
\subsection{12}
\input{2023/vector12-1.tex}
\section{2022}
\subsection{10}
\input{2022/maths1.tex}
\subsection{12}
\begin{enumerate}
\section{\textbf{Vectors}}
\item Find a unit vector perpendicular to both the vectors $\overrightarrow{a}$ and $\overrightarrow{b}$, where $\overrightarrow{a}= \hat{i} - 7\hat{j} + 7\hat{k}$ and $\overrightarrow{b} = 3\hat{i} - 2\hat{j} + 2\hat{k}$.
\item If a line has the direction ratios $-18, 12, -4$, then what are its direction cosines?
\item Show that the vectors $\hat{i} - 2\hat{j} + 3\hat{k}, - 2\hat{i} + 3\hat{j} - 4\hat{k} $ and $\hat{i} - 3\hat{j} + 5\hat{k}$ are coplanar.
\item Let $\overrightarrow{a},\overrightarrow{b}$ and $\overrightarrow{c}$ be three vectors such that $|\overrightarrow{a}| = 1$, $|\overrightarrow{b}| = 2$ and $|\overrightarrow{c}| = 3.$ If the projection of $\overrightarrow{b}$ along $\overrightarrow{a}$ is equal to the projection of $\overrightarrow{c}$ along $\overrightarrow{a}$ ; and $\overrightarrow{b}$,$\overrightarrow{c}$ are perpendicular to each other ,then find $|3\overrightarrow{a} - 2\overrightarrow{b} + 2\overrightarrow{c}|.$
\item Find the vector and cartesian equations of the plane passing through the points (2, 5, – 3), (– 2, – 3, 5) and (5, 3, – 3). Also, find the point of intersection of this plane with the line passing through points (3, 1, 5) and (– 1, – 3, – 1).
\item Find the equation of the plane passing through the intersection of the planes $\overrightarrow{r} . (\hat{i} +\hat{j} +\hat{k}) = 1$ and $\overrightarrow{r} . (2\hat{i} + 3\hat{j} - \hat{k}) + 4 = 0 $ and parallel to x-axis. Hence, find the distance of the plane from x-axis.
\end{enumerate}

\section{2021}
\subsection{10}
\input{2021/Vectors-10}
\subsection{12}
\input{2021/vectors21-12.tex}
\section{2019}
\subsection{12}
\input{2019/vectors_19.tex}
\input{2019/vect55.tex}
\input{2019/vect202.tex}
\input{2019/vect19d.tex}
\section{2019}
\subsection{10}
\input{2019/vecj.tex}
\section{2018}
\subsection{10}
\input{2018/vectors-CBSE.tex}






\chapter{Linear Forms}
\section{2023}
\subsection{10}
\input{2023/linear-10th.tex}
\subsection{12}                                                                                                  
\input{2023/linear-12th.tex}
\section{2022}
\input{2022/lin.tex}
\subsection{12}
\input{2022/LF.tex}
\section{2021}
\subsection{10}
\input{2021/linearforms.tex}
\subsection{12}
\input{2021/lin21.tex}
\section{2019}
\subsection{12}
\input{2019/linear_19.tex}
\input{2019/linearforms19d.tex}
\section{2019}
\subsection{10}
\input{2019/linearj.tex}









\chapter{Circles}
\section{2022}
\subsection{10}
\input{2022/circles.tex}
\section{2023}
\subsection{10}
\input{2023/Circle10.tex}
\section{2022}
\input{2022/tangent1.tex}
\section{2021}
\subsection{10}
\input{2021/circle-10.tex}
\section{2020}
\subsection{10}
\input{2020/circ10.tex}
\section{2019} 
\subsection{10}
\input{2019/cirj.tex}
\section{2018} 
\subsection{10}
\input{2018/circles-CBSE.tex}





\chapter{Intersection of Conics}
\section{2022}
\input{2022/chords.tex}
\section{2021}
\subsection{12}
\input{2021/con21.tex}
\section{2019}
\subsection{12}
\input{2019/intersection_19.tex}
\input{2019/inter55.tex}






\chapter{Probability}
\section{2021}
\subsection{10}
\documentclass[12pt,-letter paper]{article}                       
\usepackage{siunitx}                                              
\usepackage{setspace}
\usepackage{gensymb}                                              
\usepackage{xcolor}                                               
\usepackage{caption}
%\usepackage{subcaption}
\doublespacing                                                    
\singlespacing                                                    
\usepackage[none]{hyphenat}
\usepackage{amssymb}
\usepackage{relsize}
\usepackage[cmex10]{amsmath}
\usepackage{mathtools}
\usepackage{amsmath}                                              
\usepackage{commath}                                              
\usepackage{amsthm}
\interdisplaylinepenalty=2500
%\savesymbol{iint}
\usepackage{txfonts}                                              
%\restoresymbol{TXF}{iint}                                        
\usepackage{wasysym}                                              
\usepackage{amsthm}
\usepackage{mathrsfs}                                             
\usepackage{txfonts}                                              
\let\vec\mathbf{}
\usepackage{stfloats}
\usepackage{float}
\usepackage{cite}
\usepackage{cases}                                                
\usepackage{subfig}                                               
%\usepackage{xtab}
\usepackage{longtable}
\usepackage{multirow}
%\usepackage{algorithm}
\usepackage{amssymb}
%\usepackage{algpseudocode}
\usepackage{enumitem}
\usepackage{mathtools}
%\usepackage{eenrc}
%\usepackage[framemethod=tikz]{mdframed}                          
\usepackage{listings}                                             
%\usepackage{listings}
\usepackage[utf8]{inputenc}
%%\usepackage{color}{
%%\usepackage{lscape}
\usepackage{textcomp}
\usepackage{titling}
\usepackage{hyperref}
%\usepackage{fulbigskip}
\usepackage{tikz}
\usepackage{graphicx}                                             
\lstset{
  frame=single,
  breaklines=true
}
\let\vec\mathbf{}
\usepackage{enumitem}                                             
\usepackage{graphicx}                                             
\usepackage{siunitx}
\let\vec\mathbf{}                                         
\newcommand{\mydet}[1] 
{\ensuremath{\begin{vmatrix}
#1\end{vmatrix}}}      
\usepackage{enumitem}
\usepackage{graphicx}
\usepackage{enumitem}
\usepackage{tfrupee}
\usepackage{amsmath}
\usepackage{amssymb}
\usepackage{mwe} % for blindtext and example-image-a in example
\usepackage{wrapfig}
\usepackage{tfrupee}
\newcommand{\myvec}[1]{\ensuremath{\begin{pmatrix}#1\end{pmatrix}}}
\title{\textbf{MATHEMATIC}}
\author{Chaitanya}
\begin{document}
\maketitle
\begin{enumerate}
\section{\textbf{Probability}}
\item Mother, father and son line up at random for a family photo. If A and B 
are two events given by A = Son on one end, B = Father in the middle, 
find P(B/A).
\item Let X be a random variable which assumes values x1,x2,x3,x4 such that\\
          $2P(X = x1) = 3P(X = x2) = P(X = x3) = 5P(X = x4).$

 Find the probability distribution of X.
\item A coin is tossed 5 times. Find the probability of getting (i) at least 4 heads, and (ii) at most 4 heads.
\item There are two boxes I and II. Box I contains 3 red and 6 black balls. BoxII contains 5 red and ‘n’ black balls. One of the two boxes, box I and box II is selected at random and a ball is drawn at random. The ball drawn is found to be red. If the probability that this red ball comes out from box II is $\frac{3}{5}$, find the value of 'n'.
\end{enumerate}
\end{document}\documentclass{article}
\subsection{12}
\input{2021/probability_cbse_21.tex}
\section{2023}
\subsection{10}
\input{2023/prob.tex}
\subsection{12}
\input{2023/probability12.tex}
\section{2022}
\subsection{10}
\input{2022/probability10.tex}
\section{2022}
\subsection{12}
\documentclass[12pt,-letter paper]{article}                       
\usepackage{siunitx}                                              
\usepackage{setspace}
\usepackage{gensymb}                                              
\usepackage{xcolor}                                               
\usepackage{caption}
%\usepackage{subcaption}
\doublespacing                                                    
\singlespacing                                                    
\usepackage[none]{hyphenat}
\usepackage{amssymb}
\usepackage{relsize}
\usepackage[cmex10]{amsmath}
\usepackage{mathtools}
\usepackage{amsmath}                                              
\usepackage{commath}                                              
\usepackage{amsthm}
\interdisplaylinepenalty=2500
%\savesymbol{iint}
\usepackage{txfonts}                                              
%\restoresymbol{TXF}{iint}                                        
\usepackage{wasysym}                                              
\usepackage{amsthm}
\usepackage{mathrsfs}                                             
\usepackage{txfonts}                                              
\let\vec\mathbf{}
\usepackage{stfloats}
\usepackage{float}
\usepackage{cite}
\usepackage{cases}                                                
\usepackage{subfig}                                               
%\usepackage{xtab}
\usepackage{longtable}
\usepackage{multirow}
%\usepackage{algorithm}
\usepackage{amssymb}
%\usepackage{algpseudocode}
\usepackage{enumitem}
\usepackage{mathtools}
%\usepackage{eenrc}
%\usepackage[framemethod=tikz]{mdframed}                          
\usepackage{listings}                                             
%\usepackage{listings}
\usepackage[utf8]{inputenc}
%%\usepackage{color}{
%%\usepackage{lscape}
\usepackage{textcomp}
\usepackage{titling}
\usepackage{hyperref}
%\usepackage{fulbigskip}
\usepackage{tikz}
\usepackage{graphicx}                                             
\lstset{
  frame=single,
  breaklines=true
}
\let\vec\mathbf{}
\usepackage{enumitem}                                             
\usepackage{graphicx}                                             
\usepackage{siunitx}
\let\vec\mathbf{}                                         
\newcommand{\mydet}[1] 
{\ensuremath{\begin{vmatrix}
#1\end{vmatrix}}}      
\usepackage{enumitem}
\usepackage{graphicx}
\usepackage{enumitem}
\usepackage{tfrupee}
\usepackage{amsmath}
\usepackage{amssymb}
\usepackage{mwe} % for blindtext and example-image-a in example
\usepackage{wrapfig}
\usepackage{tfrupee}
\newcommand{\myvec}[1]{\ensuremath{\begin{pmatrix}#1\end{pmatrix}}}
\title{\textbf{MATHEMATIC}}
\author{Chaitanya}
\begin{document}
\maketitle
\begin{enumerate}
\section{\textbf{Probability}}
\item Mother, father and son line up at random for a family photo. If A and B 
are two events given by A = Son on one end, B = Father in the middle, 
find P(B/A).
\item Let X be a random variable which assumes values x1,x2,x3,x4 such that\\
          $2P(X = x1) = 3P(X = x2) = P(X = x3) = 5P(X = x4).$

 Find the probability distribution of X.
\item A coin is tossed 5 times. Find the probability of getting (i) at least 4 heads, and (ii) at most 4 heads.
\item There are two boxes I and II. Box I contains 3 red and 6 black balls. BoxII contains 5 red and ‘n’ black balls. One of the two boxes, box I and box II is selected at random and a ball is drawn at random. The ball drawn is found to be red. If the probability that this red ball comes out from box II is $\frac{3}{5}$, find the value of 'n'.
\end{enumerate}
\end{document}\documentclass{article}
\section{2020}
\subsection{10}
\input{2020/prob10.tex}
\subsection{12}
\input{2020/prob12.tex}
\section{2019}
\subsection{12}
\input{2019/probab_19.tex}
\input{2019/probb55.tex}
\input{2019/prob202.tex}
\input{2019/probab19d.tex}
\section{2019}
\subsection{10}
\input{2019/probj.tex}
\section{2018}
\subsection{10}
\input{2018/probability-CBSE.tex}





\chapter{Construction}
%\subsection{9}
\input{2023/construction-10th.tex}
\section{2022}
\subsection{10}
\input{2022/construction.tex}
\section{2021}
\subsection{10}
\input{2021/construction.tex}
\section{2020}
\subsection{10}
\input{2020/cont.tex}
\section{2019} 
\subsection{10}
\input{2019/consj.tex}
\section{2018} 
\subsection{10}
\input{2018/construction-CBSE.tex}






\chapter{Optimization}
\section{2023}
\input{2023/opti.tex}
\section{2021}
\subsection{12}
\input{2022/opt_vect.tex}
\section{2022}
\subsection{12}
\input{2021/opti-21-12.tex}
\section{2020}
\subsection{12}
\input{2020/Assignment2.tex}
\section{2019}
\subsection{12}
\input{2019/opti_19.tex}
\input{2019/opt55.tex}
\input{2019/opti202.tex}


\chapter{Algebra}
\section{2020}
\subsection{10}
\input{2020/ALGEBRA-CBSE-10.tex}
\section{2020}
\subsection{12}
\input{2020/ALGEBRA-CBSE-12.tex}
\section{2023}
\subsection{10}
\input{2023/algebra.tex}
\section{2022}
\subsection{10}
\input{2022/algebra.tex}
\section{2021}
\subsection{10}
\input{2021/algebra.tex}
\input{2021/algebra2021.tex}

\section{2021}
\subsection{12}
\input{2021/Algebra12.tex}

\section{2019}
\subsection{12}
\input{2019/algeb_19.tex}
\input{2019/alger55.tex}
\section{2019} 
\subsection{10}
\input{2019/algebj.tex}
\section{2018} 
\subsection{10}
\input{2018/Algebra-CBSE.tex}





\chapter{Geometry}
\section{2023}
\subsection{10}
\input{2023/ASSIGNMENT_1.tex}
\input{2023/latexwork.tex}
\section{2022}
\subsection{10}
\input{2022/latex.tex}

\section{2021}
\subsection{10}
\input{2021/geometry2021.tex}
\section{2019}
\subsection{10}
\input{2019/geoj.tex}
\section{2018}
\subsection{10}
\input{2018/Geometry-CBSE.tex}






\section{2020}
\subsection{10}
\input{2020/Gupdates_10.tex}

%\input{2023/gouthami.tex}
%\input{2023/algebra-10.tex}
%\subsection{10}
%\input{2023/bindhu.tex}

\chapter{Discrete}
\section{2022}
\subsection{10}
\input{2022/discrete.tex}
\section{2023}
\subsection{10}
\input{2023/firstlatex.tex}
\section{2021}
\subsection{10}
\input{2021/discrete_21.tex}
\section{2020}
\subsection{10}
\input{2020/dist.tex}
\section{2019}
\subsection{10}
\input{2019/dissj.tex}
\section{2018}
\subsection{10}
\input{2018/discrete-CBSE.tex
}

\chapter{Number Systems}
\section{2019}
\subsection{10}
\input{2019/numsysj.tex}



\chapter{Differentiation}
\section{2023}
\subsection{12}
\input{2023/differentiation.tex}

\section{2022}
\subsection{12}
\input{2022/maths.tex}
\section{2021}
\subsection{10}
\input{2021/diff.tex}
\section{2021}
\subsection{12}
\input{2021/diffn.tex}
\section{2021}
\subsection{12}
\input{2021/differ.tex}
\section{2020}
\subsection{12}
\input{2020/diff.tex}
\section{2019}
\subsection{12}
\input{2019/differ_19.tex}
\input{2019/differ55.tex}
\input{2019/diff202.tex}
\input{2019/differ19d.tex

\chapter{Integration}
\section{2023}
\subsection{12}
\input{2023/integration.tex}
\section{2022}
\subsection{12}
\input{2022/integration12.tex}

\section{2021}
\subsection{12}
\input{2021/Int.tex}
\section{2020}
\subsection{12}
\input{2020/Int.tex}
\section{2019}
\subsection{12}
\input{2019/int_19.tex}
\input{2019/intr55.tex}
\input{2019/inte202.tex}
\input{2019/integ19d.tex}


\chapter{Functions}
\section{2023}
\subsection{12}
\input{2023/Functions.tex}
\section{2022}
\subsection{12}
\input{2022/fun.tex}
\section{2021}
\subsection{10}
\input{2021/assignment_10.tex}
\subsection{12}
\input{2021/assignment_12.tex}
\section{2020} 
\subsection{12} 
\documentclass[12pt,-letter paper]{article}                       
\usepackage{siunitx}                                              
\usepackage{setspace}
\usepackage{gensymb}                                              
\usepackage{xcolor}                                               
\usepackage{caption}
%\usepackage{subcaption}
\doublespacing                                                    
\singlespacing                                                    
\usepackage[none]{hyphenat}
\usepackage{amssymb}
\usepackage{relsize}
\usepackage[cmex10]{amsmath}
\usepackage{mathtools}
\usepackage{amsmath}                                              
\usepackage{commath}                                              
\usepackage{amsthm}
\interdisplaylinepenalty=2500
%\savesymbol{iint}
\usepackage{txfonts}                                              
%\restoresymbol{TXF}{iint}                                        
\usepackage{wasysym}                                              
\usepackage{amsthm}
\usepackage{mathrsfs}                                             
\usepackage{txfonts}                                              
\let\vec\mathbf{}
\usepackage{stfloats}
\usepackage{float}
\usepackage{cite}
\usepackage{cases}                                                
\usepackage{subfig}                                               
%\usepackage{xtab}
\usepackage{longtable}
\usepackage{multirow}
%\usepackage{algorithm}
\usepackage{amssymb}
%\usepackage{algpseudocode}
\usepackage{enumitem}
\usepackage{mathtools}
%\usepackage{eenrc}
%\usepackage[framemethod=tikz]{mdframed}                          
\usepackage{listings}                                             
%\usepackage{listings}
\usepackage[utf8]{inputenc}
%%\usepackage{color}{
%%\usepackage{lscape}
\usepackage{textcomp}
\usepackage{titling}
\usepackage{hyperref}
%\usepackage{fulbigskip}
\usepackage{tikz}
\usepackage{graphicx}                                             
\lstset{
  frame=single,
  breaklines=true
}
\let\vec\mathbf{}
\usepackage{enumitem}                                             
\usepackage{graphicx}                                             
\usepackage{siunitx}
\let\vec\mathbf{}                                         
\newcommand{\mydet}[1] 
{\ensuremath{\begin{vmatrix}
#1\end{vmatrix}}}      
\usepackage{enumitem}
\usepackage{graphicx}
\usepackage{enumitem}
\usepackage{tfrupee}
\usepackage{amsmath}
\usepackage{amssymb}
\usepackage{mwe} % for blindtext and example-image-a in example
\usepackage{wrapfig}
\usepackage{tfrupee}
\newcommand{\myvec}[1]{\ensuremath{\begin{pmatrix}#1\end{pmatrix}}}
\title{\textbf{MATHEMATIC}}
\author{Chaitanya}
\begin{document}
\maketitle
\begin{enumerate}
\item Let $*$ be a binary operation on $\mathbb{R} - \{-1\}$ defined by\\    
$a * b = \frac{a}{b + 1}$for all $a, b \in \mathbb{R} - \{-1\}$.
                                                                                                
show that $*$ is neither commutative nor associative in ${R} - \{-1\}$
\item Show that the relation $R$ on the set $Z$ of all integers, given by $R = \{(a, b):2 \text{ divides } (a - b)\}$, is an equivalence relation.
\item If $f(x) = \frac{4x + 3}{6x - 4}$, $x \neq \frac{2}{3}$,Show that $fof(x) = x$ for all $x \neq \frac{2}{3}$. Also, find the inverse of $f$.
\end{enumerate}
\end{document}\documentclass{article}
\section{2019} 
\subsection{12}
\input{2019/functions_19.tex}
\input{2019/func55.tex}
\input{2019/func202.tex}
\input{2019/function19d.tex}

\chapter{Matrices}
\section{2020}
\subsection{10}
\input{2020/mat10.tex}
\section{2020}
\subsection{12}
\input{2020/mat12.tex}
\section{2022}
\subsection{10}
\input{2022/matrices10.tex}
\subsection{12}
\documentclass[12pt,-letter paper]{article}                       
\usepackage{siunitx}                                              
\usepackage{setspace}
\usepackage{gensymb}                                              
\usepackage{xcolor}                                               
\usepackage{caption}
%\usepackage{subcaption}
\doublespacing                                                    
\singlespacing                                                    
\usepackage[none]{hyphenat}
\usepackage{amssymb}
\usepackage{relsize}
\usepackage[cmex10]{amsmath}
\usepackage{mathtools}
\usepackage{amsmath}                                              
\usepackage{commath}                                              
\usepackage{amsthm}
\interdisplaylinepenalty=2500
%\savesymbol{iint}
\usepackage{txfonts}                                              
%\restoresymbol{TXF}{iint}                                        
\usepackage{wasysym}                                              
\usepackage{amsthm}
\usepackage{mathrsfs}                                             
\usepackage{txfonts}                                              
\let\vec\mathbf{}
\usepackage{stfloats}
\usepackage{float}
\usepackage{cite}
\usepackage{cases}                                                
\usepackage{subfig}                                               
%\usepackage{xtab}
\usepackage{longtable}
\usepackage{multirow}
%\usepackage{algorithm}
\usepackage{amssymb}
%\usepackage{algpseudocode}
\usepackage{enumitem}
\usepackage{mathtools}
%\usepackage{eenrc}
%\usepackage[framemethod=tikz]{mdframed}                          
\usepackage{listings}                                             
%\usepackage{listings}
\usepackage[utf8]{inputenc}
%%\usepackage{color}{
%%\usepackage{lscape}
\usepackage{textcomp}
\usepackage{titling}
\usepackage{hyperref}
%\usepackage{fulbigskip}
\usepackage{tikz}
\usepackage{graphicx}                                             
\lstset{
  frame=single,
  breaklines=true
}
\let\vec\mathbf{}
\usepackage{enumitem}                                             
\usepackage{graphicx}                                             
\usepackage{siunitx}
\let\vec\mathbf{}                                         
\newcommand{\mydet}[1] 
{\ensuremath{\begin{vmatrix}
#1\end{vmatrix}}}      
\usepackage{enumitem}
\usepackage{graphicx}
\usepackage{enumitem}
\usepackage{tfrupee}
\usepackage{amsmath}
\usepackage{amssymb}
\usepackage{mwe} % for blindtext and example-image-a in example
\usepackage{wrapfig}
\usepackage{tfrupee}
\newcommand{\myvec}[1]{\ensuremath{\begin{pmatrix}#1\end{pmatrix}}}
\title{\textbf{MATHEMATIC}}
\author{Chaitanya}
\begin{document}
\maketitle
\begin{enumerate}
\section{\textbf{Matrix}}
\item If $A$ is a square matrix of order 3 with $|A| = 4$, then write the value of $|-2A|$.
\item if $A =\myvec{
-3 & 6\\
-2 & 4\\
	}$ ,then show that $A^3=A$.
 \item Using properties of determinants, prove that
\begin{align*}
    \mydet{a^{2} + 1 & ab & ac\\
                ab & b^{2} + 1 & bc\\
		ac & bc & c^{2} + 1} = 1 + a^{2} + b^{2} + c^{2}
\end{align*}
\item If $A = \myvec{
		1 & -1 & 1\\
		2 & -1 & 0\\
		1 & 0 & 0
	}$, find $ A^{2}$ and show that $A^{2}
 = A^{-1}$.	
\item Using matrix method,solve the following system of equations:
	\begin{align*}
		2x - 3y + 5z &= 13\\
		3x + 2y - 4z &= -2\\
		x + y - 2z &= -2
	\end{align*}
 \end{enumerate}
\end{document}\documentclass{article}
\section{2023}
\subsection{10}
\input{2023/matrix_class_10.tex}
\subsection{12}
\input{2023/matrix_class_12.tex}
\section{2021}
\subsection{12}
\input{2021/matrix_12_2.tex}
\section{2021}
\subsection{12}
\input{2021/matrix_12_1.tex}
\subsection{10}
\input{2021/matrix_10.tex}
\section{2021}
\subsection{12}
\input{2021/matrix_12_3.tex}
\section{2019}
\subsection{12}
\input{2019/matrices_19.tex}
\input{2019/matr55.tex}
\input{2019/matr202.tex}
\input{2019/matrix19d.tex}


\section{2019}
\subsection{10}
\input{2019/matrrj.tex}



\chapter{Trignometry}
\section{2019}
\subsection{10}
\input{2019/trignj.tex}


%
%\include{ch02} 
\backmatter
\appendix
\iffalse
\chapter{Conic Lines}
\section{Pair of Straight Lines}
%
\input{quad/pair.tex}
\section{Intersection of Conics}
\input{quadlines/inter.tex}
\section{ Chords of a Conic}
\input{quadlines/chord.tex}
\section{ Tangent and Normal}
\input{quadlines/tangent.tex}
\fi
%\chapter{Proofs}
%   \section{}
%\input{apps/defs.tex}

%  \section{}
%\input{apps/parab.tex}
%  \section{}
%\input{apps/nonparab.tex}
%		\section{}
%\input{apps/params.tex}
\latexprintindex

\end{document}

 
\section{Examples}
\subsection{Loney}
\input{examples/loney.tex}
\subsection{Miscellaneous}
\input{examples/misc.tex}
%
%%\section*{Disclosure Statement}
%%The authors report there are no competing interests to declare.
%%
%%
%%
%%  
%%%All the results related to conics are summarized in 
%%%Table \ref{table:conics}.  
%%%\begin{table*}[!t]
%%%\centering
%%%\input{conics.tex}
%%%%\input{./figs/conics.tex}
%%%\caption{$\vec{x}^{\top}\vec{V}\vec{x}+2\vec{u}^{\top}\vec{x}+f = 0$  can be expressed in the above standard form for various conics. $\vec{c}$ represents the centre/vertex of the conic. $\vec{q}$ is/are the point(s) of contact for the tangent(s). }
%%%\label{table:conics}
%%%\end{table*}
%%%\begin{verbatim}
%%\bibliographystyle{tfs}
%%%\bibliography{interacttfssample}
%%\bibliography{school}
%%\end{verbatim}
%% included where the list of references is to appear, where \texttt{tfs.bst} is the name of the \textsc{Bib}\TeX\ bibliography style file for Taylor \& Francis' Reference Style S and \texttt{interacttfssample.bib} is the bibliographic database included with the \textsf{Interact}-TFS \LaTeX\ bundle (to be replaced with the name of your own .bib file). \LaTeX/\textsc{Bib}\TeX\ will extract from your .bib file only those references that are cited in your .tex file and list them in the References section.
%
%% Please include a copy of your .bib file and/or the final generated .bbl file among your source files if your .tex file does not contain a reference list in a \texttt{thebibliography} environment.
%

  % \section{Appendices}
  % \appendix
			\appendices

\documentclass[12pt,-letter paper]{article}                       
\usepackage{siunitx}                                              
\usepackage{setspace}
\usepackage{gensymb}                                              
\usepackage{xcolor}                                               
\usepackage{caption}
%\usepackage{subcaption}
\doublespacing                                                    
\singlespacing                                                    
\usepackage[none]{hyphenat}
\usepackage{amssymb}
\usepackage{relsize}
\usepackage[cmex10]{amsmath}
\usepackage{mathtools}
\usepackage{amsmath}                                              
\usepackage{commath}                                              
\usepackage{amsthm}
\interdisplaylinepenalty=2500
%\savesymbol{iint}
\usepackage{txfonts}                                              
%\restoresymbol{TXF}{iint}                                        
\usepackage{wasysym}                                              
\usepackage{amsthm}
\usepackage{mathrsfs}                                             
\usepackage{txfonts}                                              
\let\vec\mathbf{}
\usepackage{stfloats}
\usepackage{float}
\usepackage{cite}
\usepackage{cases}                                                
\usepackage{subfig}                                               
%\usepackage{xtab}
\usepackage{longtable}
\usepackage{multirow}
%\usepackage{algorithm}
\usepackage{amssymb}
%\usepackage{algpseudocode}
\usepackage{enumitem}
\usepackage{mathtools}
%\usepackage{eenrc}
%\usepackage[framemethod=tikz]{mdframed}                          
\usepackage{listings}                                             
%\usepackage{listings}
\usepackage[utf8]{inputenc}
%%\usepackage{color}{
%%\usepackage{lscape}
\usepackage{textcomp}
\usepackage{titling}
\usepackage{hyperref}
%\usepackage{fulbigskip}
\usepackage{tikz}
\usepackage{graphicx}                                             
\lstset{
  frame=single,
  breaklines=true
}
\let\vec\mathbf{}
\usepackage{enumitem}                                             
\usepackage{graphicx}                                             
\usepackage{siunitx}
\let\vec\mathbf{}                                         
\newcommand{\mydet}[1] 
{\ensuremath{\begin{vmatrix}
#1\end{vmatrix}}}      
\usepackage{enumitem}
\usepackage{graphicx}
\usepackage{enumitem}
\usepackage{tfrupee}
\usepackage{amsmath}
\usepackage{amssymb}
\usepackage{mwe} % for blindtext and example-image-a in example
\usepackage{wrapfig}
\usepackage{tfrupee}
\newcommand{\myvec}[1]{\ensuremath{\begin{pmatrix}#1\end{pmatrix}}}
\title{\textbf{MATHEMATIC}}
\author{Chaitanya}
\begin{document}
\maketitle
\begin{enumerate}
\section{\textbf{Vectors}}
\item Find a unit vector perpendicular to both the vectors $\overrightarrow{a}$ and $\overrightarrow{b}$, where $\overrightarrow{a}= \hat{i} - 7\hat{j} + 7\hat{k}$ and $\overrightarrow{b} = 3\hat{i} - 2\hat{j} + 2\hat{k}$.
\item If a line has the direction ratios $-18, 12, -4$, then what are its direction cosines?
\item Show that the vectors $\hat{i} - 2\hat{j} + 3\hat{k}, - 2\hat{i} + 3\hat{j} - 4\hat{k} $ and $\hat{i} - 3\hat{j} + 5\hat{k}$ are coplanar.
\item Let $\overrightarrow{a},\overrightarrow{b}$ and $\overrightarrow{c}$ be three vectors such that $|\overrightarrow{a}| = 1$, $|\overrightarrow{b}| = 2$ and $|\overrightarrow{c}| = 3.$ If the projection of $\overrightarrow{b}$ along $\overrightarrow{a}$ is equal to the projection of $\overrightarrow{c}$ along $\overrightarrow{a}$ ; and $\overrightarrow{b}$,$\overrightarrow{c}$ are perpendicular to each other ,then find $|3\overrightarrow{a} - 2\overrightarrow{b} + 2\overrightarrow{c}|.$
\item Find the vector and cartesian equations of the plane passing through the points (2, 5, – 3), (– 2, – 3, 5) and (5, 3, – 3). Also, find the point of intersection of this plane with the line passing through points (3, 1, 5) and (– 1, – 3, – 1).
\item Find the equation of the plane passing through the intersection of the planes $\overrightarrow{r} . (\hat{i} +\hat{j} +\hat{k}) = 1$ and $\overrightarrow{r} . (2\hat{i} + 3\hat{j} - \hat{k}) + 4 = 0 $ and parallel to x-axis. Hence, find the distance of the plane from x-axis.
\end{enumerate}
\end{document}\documentclass{article}